
\item Does it pay to take preparatory courses for standardised tests such as the Comptia Exams? 

Using the sample data in the following table, compute the case-wise differences, the mean of the case-wise differences and the standard deviation of the case wise differences for the following data set.

\begin{center}
	\begin{tabular}{|c|c|c|c|c|c|c|c|c|c|c|}
		\hline  
		Student	&	A	&	B	&	C	&	D	&	E	&	F	&	G	&	H	&	I	&	J	\\ \hline
		Score Before	&	700	&	840	&	830	&	860	&	840	&	690	&	830	&	1180	&	930	&	1070	\\ \hline
		Score After	&	720	&	840	&	820	&	900	&	870	&	700	&	800	&	1200	&	950	&	1080	\\ \hline
	\end{tabular} 
\end{center}

### **Does Taking Preparatory Courses Improve Exam Performance?**

To analyze the effect of preparatory courses on **standardized test scores**, we compute:

1. **Case-wise Differences:** Difference between "Score After" and "Score Before" for each student.
2. **Mean of the Case-wise Differences:** Average improvement across students.
3. **Standard Deviation of the Case-wise Differences:** Measures variability in score improvements.

---

### **Step 1: Compute Case-wise Differences**
\[
d_i = \text{Score After} - \text{Score Before}
\]

| Student | Score Before | Score After | \( d_i \) (Difference) |
|---------|-------------|-------------|------------------|
| A       | 700        | 720        | 20              |
| B       | 840        | 840        | 0               |
| C       | 830        | 820        | -10             |
| D       | 860        | 900        | 40              |
| E       | 840        | 870        | 30              |
| F       | 690        | 700        | 10              |
| G       | 830        | 800        | -30             |
| H       | 1180       | 1200       | 20              |
| I       | 930        | 950        | 20              |
| J       | 1070       | 1080       | 10              |

---

### **Step 2: Compute Mean Difference**
\[
\bar{d} = \frac{\sum d_i}{n}
\]

\[
\bar{d} = \frac{20 + 0 - 10 + 40 + 30 + 10 - 30 + 20 + 20 + 10}{10} = \frac{110}{10} = 11
\]

So, the **mean improvement** across students is **11 points**.

---

### **Step 3: Compute Standard Deviation**
\[
s_d = \sqrt{ \frac{\sum d_i^2 - n\bar{d}^2}{n-1} }
\]

#### **Compute \( \sum d_i^2 \):**
\[
\sum d_i^2 = (20^2 + 0^2 + (-10)^2 + 40^2 + 30^2 + 10^2 + (-30)^2 + 20^2 + 20^2 + 10^2)
\]

\[
= (400 + 0 + 100 + 1600 + 900 + 100 + 900 + 400 + 400 + 100) = 5900
\]

#### **Compute \( s_d \):**
\[
s_d = \sqrt{ \frac{5900 - 10(11)^2}{9} }
\]

\[
s_d = \sqrt{ \frac{5900 - 1210}{9} }
\]

\[
s_d = \sqrt{ \frac{4690}{9} } = \sqrt{521.11} \approx 22.83
\]

---

### **Conclusion**
- The **mean improvement** is **11 points**, indicating **some** effect of preparatory courses.
- The **standard deviation** is **22.83**, suggesting **variation** in score improvement across students.
- While some students **gained significantly**, others **showed no improvement** or even a **slight decline**.

Would you like to interpret these results further or conduct additional statistical tests? 🚀
