	\documentclass[a4paper,12pt]{article}
%%%%%%%%%%%%%%%%%%%%%%%%%%%%%%%%%%%%%%%%%%%%%%%%%%%%%%%%%%%%%%%%%%%%%%%%%%%%%%%%%%%%%%%%%%%%%%%%%%%%%%%%%%%%%%%%%%%%%%%%%%%%%%%%%%%%%%%%%%%%%%%%%%%%%%%%%%%%%%%%%%%%%%%%%%%%%%%%%%%%%%%%%%%%%%%%%%%%%%%%%%%%%%%%%%%%%%%%%%%%%%%%%%%%%%%%%%%%%%%%%%%%%%%%%%%%
\usepackage{eurosym}
\usepackage{vmargin}
\usepackage{amsmath}
\usepackage{framed}
\usepackage{graphics}
\usepackage{epsfig}
\usepackage{subfigure}
\usepackage{enumerate}
\usepackage{fancyhdr}

\setcounter{MaxMatrixCols}{10}
%TCIDATA{OutputFilter=LATEX.DLL}
%TCIDATA{Version=5.00.0.2570}
%TCIDATA{<META NAME="SaveForMode"CONTENT="1">}
%TCIDATA{LastRevised=Wednesday, February 23, 201113:24:34}
%TCIDATA{<META NAME="GraphicsSave" CONTENT="32">}
%TCIDATA{Language=American English}

\pagestyle{fancy}
\setmarginsrb{20mm}{0mm}{20mm}{25mm}{12mm}{11mm}{0mm}{11mm}
\lhead{Statistics with \texttt{r}} \rhead{Kevin O'Brien} \chead{Chi-Square Test} %\input{tcilatex}

\begin{document}

%----------------------------------------------------%

\subsubsection{Chi-squared Test}

A $\chi^2$ test is carried out on tabular data containing counts, e.g. the
number of animals that died, the number of days of rain, the
number of stocks that grew in value, etc.

Usually have two qualitative variables, each with a number of
levels, and want to determine if there is a relationship between the
two variables, e.g. hair colour and eye colour, social status and
crime rates, house price and house size, gender and left/right
handedness.

The data are presented in a contingency table:
right-handed left-handed TOTAL

\begin{tabular}{|c|c|c|c|}
	\hline
	% after \\: \hline or \cline{col1-col2} \cline{col3-col4} ...
	& right-handed &left-handed & TOTAL\\\hline
	Male & 43 & 9 & 52 \\
	Female & 44 & 4 & 48 \\
	TOTAL & 87 & 13 & 100 \\
	\hline
\end{tabular}


The hypothesis to be tested is
$H0 :$There is no relationship between gender and left/right-handedness
$H1 :$There is a relationship between gender and left/right-handedness
The values that we collect from our sample are called the observed
(O) frequencies (counts). Now need to calculate the expected (E)
frequencies, i.e. the values we would expect to see in the table, if
H0 was true.


	%========================================================================================%
	
	\subsection{Chi-Square Test}
	
	% \vspace{-1cm}
	In MASS R package, we can find the data set \textbf{survey}. The \textbf{Smoke} column records the students smoking habit, while the 
	\textbf{Exer} column records their exercise level. 
	\begin{framed}
		\begin{verbatim}
		library(MASS)
		
		head(survey)
		names(survey)
		\end{verbatim}
	\end{framed}
	
	%========================================================================================%
	
	\subsection{Chi-Square Test}
	
	% \vspace{-1cm}
	\textbf{Levels}
	\begin{itemize}
		\item These variables are factors (i.e. categorical variables).
		\bigskip
		\item The allowed values in \textbf{Smoke} are \textit{Heavy}, \textit{Regul} (regularly), \textit{Occas} (occasionally) and \textit{Never}. 
		\bigskip
		\item As for \textbf{Exer}, they are \textit{Freq} (frequently), \textit{Some} and \textit{None}.
	\end{itemize}
	
	%========================================================================================%
	
	\subsection{Chi-Square Test}
	
	% \vspace{-1cm}
	\begin{itemize}
		\item We can construct the contingency table of the two variables using the \texttt{table()} function.
		\bigskip
		\item This table will compare the students smoking habit across the various exercise levels. 
	\end{itemize}
	
	
	%========================================================================================%
	
	\subsection{Chi-Square Test}
	
	\textbf{Solution}
	\begin{framed}
		\begin{verbatim}
		> # load the MASS package 
		> library(MASS)       
		> myTable= table(survey$Smoke, survey$Exer) 
		\end{verbatim}
	\end{framed}
	
	
	%========================================================================================%
	
	
	\textbf{Contingency Table }
	\begin{framed}
		\begin{verbatim}
		>    
		Freq None Some 
		Heavy    7    1    3 
		Never   87   18   84 
		Occas   12    3    4 
		Regul    9    1    7
		\end{verbatim}
	\end{framed}
	Small Cell Sizes - This can cause problems. 
	
	
	%========================================================================================%
	
	
	\textbf{Contingency Table }
	
	
	\begin{itemize}
		\item \textbf{Exer} the levels are \textit{Freq} (frequently), \textit{Some} and \textit{None}.
		\item Combine \textit{Some} and \textit{None} into a  single "not frequently" level.
		\item Preserve original data, create new object called \textbf{Exer2}.
	\end{itemize}
	

	\textbf{Problem}\\
	\begin{itemize}
		\item Test the hypothesis whether the students smoking habit is independent of their exercise level at 0.05 significance level.
	\end{itemize}
	
	% \vspace{-1cm}
	\textbf{Solution}\\
	\begin{itemize}
		\item We apply the \texttt{chisq.test} function to the data in the contingency table \texttt{myTable}.
		\item The p-value was found to 0.4828.
	\end{itemize}
	
	
	%========================================================================================%
	
	
	\textbf{Solution}\\
	\begin{itemize}
		\item The p-value 0.4828 is greater than the 0.05 significance level \bigskip
		\item Therefore, we do not reject the null hypothesis that the smoking habit is independent of the exercise level of the students.
	\end{itemize}
	
	%========================================================================================%

\end{document}
